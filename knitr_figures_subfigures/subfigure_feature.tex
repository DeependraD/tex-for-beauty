\documentclass[]{article}
\usepackage{lmodern}
\usepackage{amssymb,amsmath}
\usepackage{ifxetex,ifluatex}
\usepackage{fixltx2e} % provides \textsubscript
\ifnum 0\ifxetex 1\fi\ifluatex 1\fi=0 % if pdftex
  \usepackage[T1]{fontenc}
  \usepackage[utf8]{inputenc}
\else % if luatex or xelatex
  \ifxetex
    \usepackage{mathspec}
  \else
    \usepackage{fontspec}
  \fi
  \defaultfontfeatures{Ligatures=TeX,Scale=MatchLowercase}
\fi
% use upquote if available, for straight quotes in verbatim environments
\IfFileExists{upquote.sty}{\usepackage{upquote}}{}
% use microtype if available
\IfFileExists{microtype.sty}{%
\usepackage{microtype}
\UseMicrotypeSet[protrusion]{basicmath} % disable protrusion for tt fonts
}{}
\usepackage[margin=1in]{geometry}
\usepackage{hyperref}
\hypersetup{unicode=true,
            pdfborder={0 0 0},
            breaklinks=true}
\urlstyle{same}  % don't use monospace font for urls
\usepackage{color}
\usepackage{fancyvrb}
\newcommand{\VerbBar}{|}
\newcommand{\VERB}{\Verb[commandchars=\\\{\}]}
\DefineVerbatimEnvironment{Highlighting}{Verbatim}{commandchars=\\\{\}}
% Add ',fontsize=\small' for more characters per line
\usepackage{framed}
\definecolor{shadecolor}{RGB}{248,248,248}
\newenvironment{Shaded}{\begin{snugshade}}{\end{snugshade}}
\newcommand{\KeywordTok}[1]{\textcolor[rgb]{0.13,0.29,0.53}{\textbf{#1}}}
\newcommand{\DataTypeTok}[1]{\textcolor[rgb]{0.13,0.29,0.53}{#1}}
\newcommand{\DecValTok}[1]{\textcolor[rgb]{0.00,0.00,0.81}{#1}}
\newcommand{\BaseNTok}[1]{\textcolor[rgb]{0.00,0.00,0.81}{#1}}
\newcommand{\FloatTok}[1]{\textcolor[rgb]{0.00,0.00,0.81}{#1}}
\newcommand{\ConstantTok}[1]{\textcolor[rgb]{0.00,0.00,0.00}{#1}}
\newcommand{\CharTok}[1]{\textcolor[rgb]{0.31,0.60,0.02}{#1}}
\newcommand{\SpecialCharTok}[1]{\textcolor[rgb]{0.00,0.00,0.00}{#1}}
\newcommand{\StringTok}[1]{\textcolor[rgb]{0.31,0.60,0.02}{#1}}
\newcommand{\VerbatimStringTok}[1]{\textcolor[rgb]{0.31,0.60,0.02}{#1}}
\newcommand{\SpecialStringTok}[1]{\textcolor[rgb]{0.31,0.60,0.02}{#1}}
\newcommand{\ImportTok}[1]{#1}
\newcommand{\CommentTok}[1]{\textcolor[rgb]{0.56,0.35,0.01}{\textit{#1}}}
\newcommand{\DocumentationTok}[1]{\textcolor[rgb]{0.56,0.35,0.01}{\textbf{\textit{#1}}}}
\newcommand{\AnnotationTok}[1]{\textcolor[rgb]{0.56,0.35,0.01}{\textbf{\textit{#1}}}}
\newcommand{\CommentVarTok}[1]{\textcolor[rgb]{0.56,0.35,0.01}{\textbf{\textit{#1}}}}
\newcommand{\OtherTok}[1]{\textcolor[rgb]{0.56,0.35,0.01}{#1}}
\newcommand{\FunctionTok}[1]{\textcolor[rgb]{0.00,0.00,0.00}{#1}}
\newcommand{\VariableTok}[1]{\textcolor[rgb]{0.00,0.00,0.00}{#1}}
\newcommand{\ControlFlowTok}[1]{\textcolor[rgb]{0.13,0.29,0.53}{\textbf{#1}}}
\newcommand{\OperatorTok}[1]{\textcolor[rgb]{0.81,0.36,0.00}{\textbf{#1}}}
\newcommand{\BuiltInTok}[1]{#1}
\newcommand{\ExtensionTok}[1]{#1}
\newcommand{\PreprocessorTok}[1]{\textcolor[rgb]{0.56,0.35,0.01}{\textit{#1}}}
\newcommand{\AttributeTok}[1]{\textcolor[rgb]{0.77,0.63,0.00}{#1}}
\newcommand{\RegionMarkerTok}[1]{#1}
\newcommand{\InformationTok}[1]{\textcolor[rgb]{0.56,0.35,0.01}{\textbf{\textit{#1}}}}
\newcommand{\WarningTok}[1]{\textcolor[rgb]{0.56,0.35,0.01}{\textbf{\textit{#1}}}}
\newcommand{\AlertTok}[1]{\textcolor[rgb]{0.94,0.16,0.16}{#1}}
\newcommand{\ErrorTok}[1]{\textcolor[rgb]{0.64,0.00,0.00}{\textbf{#1}}}
\newcommand{\NormalTok}[1]{#1}
\usepackage{longtable,booktabs}
\usepackage{graphicx,grffile}
\makeatletter
\def\maxwidth{\ifdim\Gin@nat@width>\linewidth\linewidth\else\Gin@nat@width\fi}
\def\maxheight{\ifdim\Gin@nat@height>\textheight\textheight\else\Gin@nat@height\fi}
\makeatother
% Scale images if necessary, so that they will not overflow the page
% margins by default, and it is still possible to overwrite the defaults
% using explicit options in \includegraphics[width, height, ...]{}
\setkeys{Gin}{width=\maxwidth,height=\maxheight,keepaspectratio}
\IfFileExists{parskip.sty}{%
\usepackage{parskip}
}{% else
\setlength{\parindent}{0pt}
\setlength{\parskip}{6pt plus 2pt minus 1pt}
}
\setlength{\emergencystretch}{3em}  % prevent overfull lines
\providecommand{\tightlist}{%
  \setlength{\itemsep}{0pt}\setlength{\parskip}{0pt}}
\setcounter{secnumdepth}{5}
% Redefines (sub)paragraphs to behave more like sections
\ifx\paragraph\undefined\else
\let\oldparagraph\paragraph
\renewcommand{\paragraph}[1]{\oldparagraph{#1}\mbox{}}
\fi
\ifx\subparagraph\undefined\else
\let\oldsubparagraph\subparagraph
\renewcommand{\subparagraph}[1]{\oldsubparagraph{#1}\mbox{}}
\fi

%%% Use protect on footnotes to avoid problems with footnotes in titles
\let\rmarkdownfootnote\footnote%
\def\footnote{\protect\rmarkdownfootnote}

%%% Change title format to be more compact
\usepackage{titling}

% Create subtitle command for use in maketitle
\newcommand{\subtitle}[1]{
  \posttitle{
    \begin{center}\large#1\end{center}
    }
}

\setlength{\droptitle}{-2em}
  \title{}
  \pretitle{\vspace{\droptitle}}
  \posttitle{}
  \author{}
  \preauthor{}\postauthor{}
  \date{}
  \predate{}\postdate{}

\usepackage{booktabs}
\usepackage{longtable}
\usepackage{array}
\usepackage{multirow}
\usepackage[table]{xcolor}
\usepackage{wrapfig}
\usepackage{float}
\usepackage{colortbl}
\usepackage{pdflscape}
\usepackage{tabu}
\usepackage{threeparttable}
\usepackage[normalem]{ulem}

\usepackage{longtable}
\usepackage{graphicx,subcaption}
\usepackage{tabularx}
\newcommand{\subfloat}[2][need a sub-caption]{\subcaptionbox{#1}{#2}}

\begin{document}

{
\setcounter{tocdepth}{2}
\tableofcontents
}
\begin{Shaded}
\begin{Highlighting}[]
\KeywordTok{plot}\NormalTok{(}\DecValTok{1}\OperatorTok{:}\DecValTok{10}\NormalTok{)}
\KeywordTok{plot}\NormalTok{(}\DecValTok{1}\OperatorTok{:}\DecValTok{10}\NormalTok{, }\DecValTok{3}\OperatorTok{:}\DecValTok{12}\NormalTok{)}
\KeywordTok{plot}\NormalTok{(}\KeywordTok{rnorm}\NormalTok{(}\DecValTok{10}\NormalTok{), }\DataTypeTok{pch=}\DecValTok{19}\NormalTok{)}
\KeywordTok{plot}\NormalTok{(}\KeywordTok{rnorm}\NormalTok{(}\DecValTok{10}\NormalTok{), }\DataTypeTok{pch=}\DecValTok{21}\NormalTok{)}
\KeywordTok{plot}\NormalTok{(}\KeywordTok{rnorm}\NormalTok{(}\DecValTok{10}\NormalTok{), }\DataTypeTok{pch=}\DecValTok{12}\NormalTok{)}
\KeywordTok{plot}\NormalTok{(}\KeywordTok{rnorm}\NormalTok{(}\DecValTok{10}\NormalTok{), }\DataTypeTok{pch=}\DecValTok{14}\NormalTok{)}
\end{Highlighting}
\end{Shaded}

\begin{figure}
\subfloat[one plot this one subcaption is supposed to be really really really very long\label{fig:fig-sub1}]{\includegraphics[width=.49\linewidth]{Subfigure_feature_files/figure-latex/fig-sub-1} }\subfloat[the other one\label{fig:fig-sub2}]{\includegraphics[width=.49\linewidth]{Subfigure_feature_files/figure-latex/fig-sub-2} }\newline\subfloat[the third one this one caption is not long enought to be wrapped around, I guess. But I could be wrong.\label{fig:fig-sub3}]{\includegraphics[width=.49\linewidth]{Subfigure_feature_files/figure-latex/fig-sub-3} }\subfloat[the fourth and the obstructive one\label{fig:fig-sub4}]{\includegraphics[width=.49\linewidth]{Subfigure_feature_files/figure-latex/fig-sub-4} }\newline\subfloat[the fifth and sadistic one\label{fig:fig-sub5}]{\includegraphics[width=.49\linewidth]{Subfigure_feature_files/figure-latex/fig-sub-5} }\subfloat[the last but not the least one\label{fig:fig-sub6}]{\includegraphics[width=.49\linewidth]{Subfigure_feature_files/figure-latex/fig-sub-6} }\caption{two plots}\label{fig:fig-sub}
\end{figure}

\small{
\begin{longtable}[]{@{}ll@{}}
\caption{\label{tab:site-fact-tab}Site of study factsheet}\\
\toprule
\textbf{Particulars} & \textbf{Detail}\tabularnewline
\midrule
\endhead
Institution & Agriculture and Forestry University\tabularnewline
Farm & Rampur\tabularnewline
State & Chitwan\tabularnewline
Longitude & \(84.4^o\)\tabularnewline
Latitude & \(27.62^o\)\tabularnewline
Altitude & 191 meters\tabularnewline
Environment & ME1\tabularnewline
Dominant Soil(FAO classification) & Cambisols (Dystric
Cambisols)\tabularnewline
Surface pH & 5\tabularnewline
Cropping system & Rice-wheat\tabularnewline
\bottomrule
\end{longtable}
}

\begingroup  \small 

\begin{tabular}{@{\extracolsep{1pt}}lc} 
\\[-1.8ex]\hline 
\hline \\[-1.8ex] 
 & \multicolumn{1}{c}{\textit{Dependent variable:}} \\ 
\cline{2-2} 
\\[-1.8ex] & \textit{OLS} \\ 
 & mpg \\ 
\hline \\[-1.8ex] 
 a & $-$0.03 (0.02) \\ 
  & p = 0.11 \\ 
  b & $-$1.61$^{*}$ (0.83) \\ 
  & p = 0.07 \\ 
  c & $-$0.68 (0.47) \\ 
  & p = 0.16 \\ 
 \hline \\[-1.8ex] 
Observations & 32 \\ 
R$^{2}$ & 0.78 \\ 
Residual Std. Error & 3.00 \\ 
F Statistic & 24.59$^{***}$  (p = 0.00) \\ 
\hline 
\hline \\[-1.8ex] 
\textit{Note:}  & \multicolumn{1}{r}{$^{*}$p$<$0.1; $^{**}$p$<$0.05; $^{***}$p$<$0.01} \\ 
\end{tabular}

\endgroup 

\begin{table}[H]
\centering
\begin{tabular}{>{\raggedright\arraybackslash}p{2.5cm}llllllll}
\toprule
term & df & AIC & BIC & logLik & deviance & statistic & Chi.Df & p.value\\
\midrule
Blocking factors & 6 & 170.1592 & 178.9536 & -79.07961 & 158.1592 & NA & NA & NA\\
Blocking factors + Entry genotypes & 7 & 172.1592 & 182.4194 & -79.07961 & 158.1592 & 0.000000 & 1 & 0.9999999\\
Blocking factors + Check and entry genotypes & 8 & 167.5311 & 179.2570 & -75.76555 & 151.5311 & 6.628114 & 1 & 0.0100381\\
\bottomrule
\end{tabular}
\end{table}


\end{document}
