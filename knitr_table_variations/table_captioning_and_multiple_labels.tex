\documentclass[]{article}
\usepackage{lmodern}
\usepackage{amssymb,amsmath}
\usepackage{ifxetex,ifluatex}
\usepackage{fixltx2e} % provides \textsubscript
\ifnum 0\ifxetex 1\fi\ifluatex 1\fi=0 % if pdftex
  \usepackage[T1]{fontenc}
  \usepackage[utf8]{inputenc}
\else % if luatex or xelatex
  \ifxetex
    \usepackage{mathspec}
  \else
    \usepackage{fontspec}
  \fi
  \defaultfontfeatures{Ligatures=TeX,Scale=MatchLowercase}
\fi
% use upquote if available, for straight quotes in verbatim environments
\IfFileExists{upquote.sty}{\usepackage{upquote}}{}
% use microtype if available
\IfFileExists{microtype.sty}{%
\usepackage{microtype}
\UseMicrotypeSet[protrusion]{basicmath} % disable protrusion for tt fonts
}{}
\usepackage[margin=1in]{geometry}
\usepackage{hyperref}
\hypersetup{unicode=true,
            pdftitle={Table edit},
            pdfauthor={Deependra Dhakal},
            pdfborder={0 0 0},
            breaklinks=true}
\urlstyle{same}  % don't use monospace font for urls
\usepackage{color}
\usepackage{fancyvrb}
\newcommand{\VerbBar}{|}
\newcommand{\VERB}{\Verb[commandchars=\\\{\}]}
\DefineVerbatimEnvironment{Highlighting}{Verbatim}{commandchars=\\\{\}}
% Add ',fontsize=\small' for more characters per line
\usepackage{framed}
\definecolor{shadecolor}{RGB}{248,248,248}
\newenvironment{Shaded}{\begin{snugshade}}{\end{snugshade}}
\newcommand{\KeywordTok}[1]{\textcolor[rgb]{0.13,0.29,0.53}{\textbf{#1}}}
\newcommand{\DataTypeTok}[1]{\textcolor[rgb]{0.13,0.29,0.53}{#1}}
\newcommand{\DecValTok}[1]{\textcolor[rgb]{0.00,0.00,0.81}{#1}}
\newcommand{\BaseNTok}[1]{\textcolor[rgb]{0.00,0.00,0.81}{#1}}
\newcommand{\FloatTok}[1]{\textcolor[rgb]{0.00,0.00,0.81}{#1}}
\newcommand{\ConstantTok}[1]{\textcolor[rgb]{0.00,0.00,0.00}{#1}}
\newcommand{\CharTok}[1]{\textcolor[rgb]{0.31,0.60,0.02}{#1}}
\newcommand{\SpecialCharTok}[1]{\textcolor[rgb]{0.00,0.00,0.00}{#1}}
\newcommand{\StringTok}[1]{\textcolor[rgb]{0.31,0.60,0.02}{#1}}
\newcommand{\VerbatimStringTok}[1]{\textcolor[rgb]{0.31,0.60,0.02}{#1}}
\newcommand{\SpecialStringTok}[1]{\textcolor[rgb]{0.31,0.60,0.02}{#1}}
\newcommand{\ImportTok}[1]{#1}
\newcommand{\CommentTok}[1]{\textcolor[rgb]{0.56,0.35,0.01}{\textit{#1}}}
\newcommand{\DocumentationTok}[1]{\textcolor[rgb]{0.56,0.35,0.01}{\textbf{\textit{#1}}}}
\newcommand{\AnnotationTok}[1]{\textcolor[rgb]{0.56,0.35,0.01}{\textbf{\textit{#1}}}}
\newcommand{\CommentVarTok}[1]{\textcolor[rgb]{0.56,0.35,0.01}{\textbf{\textit{#1}}}}
\newcommand{\OtherTok}[1]{\textcolor[rgb]{0.56,0.35,0.01}{#1}}
\newcommand{\FunctionTok}[1]{\textcolor[rgb]{0.00,0.00,0.00}{#1}}
\newcommand{\VariableTok}[1]{\textcolor[rgb]{0.00,0.00,0.00}{#1}}
\newcommand{\ControlFlowTok}[1]{\textcolor[rgb]{0.13,0.29,0.53}{\textbf{#1}}}
\newcommand{\OperatorTok}[1]{\textcolor[rgb]{0.81,0.36,0.00}{\textbf{#1}}}
\newcommand{\BuiltInTok}[1]{#1}
\newcommand{\ExtensionTok}[1]{#1}
\newcommand{\PreprocessorTok}[1]{\textcolor[rgb]{0.56,0.35,0.01}{\textit{#1}}}
\newcommand{\AttributeTok}[1]{\textcolor[rgb]{0.77,0.63,0.00}{#1}}
\newcommand{\RegionMarkerTok}[1]{#1}
\newcommand{\InformationTok}[1]{\textcolor[rgb]{0.56,0.35,0.01}{\textbf{\textit{#1}}}}
\newcommand{\WarningTok}[1]{\textcolor[rgb]{0.56,0.35,0.01}{\textbf{\textit{#1}}}}
\newcommand{\AlertTok}[1]{\textcolor[rgb]{0.94,0.16,0.16}{#1}}
\newcommand{\ErrorTok}[1]{\textcolor[rgb]{0.64,0.00,0.00}{\textbf{#1}}}
\newcommand{\NormalTok}[1]{#1}
\usepackage{longtable,booktabs}
\usepackage{graphicx,grffile}
\makeatletter
\def\maxwidth{\ifdim\Gin@nat@width>\linewidth\linewidth\else\Gin@nat@width\fi}
\def\maxheight{\ifdim\Gin@nat@height>\textheight\textheight\else\Gin@nat@height\fi}
\makeatother
% Scale images if necessary, so that they will not overflow the page
% margins by default, and it is still possible to overwrite the defaults
% using explicit options in \includegraphics[width, height, ...]{}
\setkeys{Gin}{width=\maxwidth,height=\maxheight,keepaspectratio}
\IfFileExists{parskip.sty}{%
\usepackage{parskip}
}{% else
\setlength{\parindent}{0pt}
\setlength{\parskip}{6pt plus 2pt minus 1pt}
}
\setlength{\emergencystretch}{3em}  % prevent overfull lines
\providecommand{\tightlist}{%
  \setlength{\itemsep}{0pt}\setlength{\parskip}{0pt}}
\setcounter{secnumdepth}{5}
% Redefines (sub)paragraphs to behave more like sections
\ifx\paragraph\undefined\else
\let\oldparagraph\paragraph
\renewcommand{\paragraph}[1]{\oldparagraph{#1}\mbox{}}
\fi
\ifx\subparagraph\undefined\else
\let\oldsubparagraph\subparagraph
\renewcommand{\subparagraph}[1]{\oldsubparagraph{#1}\mbox{}}
\fi

%%% Use protect on footnotes to avoid problems with footnotes in titles
\let\rmarkdownfootnote\footnote%
\def\footnote{\protect\rmarkdownfootnote}

%%% Change title format to be more compact
\usepackage{titling}

% Create subtitle command for use in maketitle
\newcommand{\subtitle}[1]{
  \posttitle{
    \begin{center}\large#1\end{center}
    }
}

\setlength{\droptitle}{-2em}
  \title{Table edit}
  \pretitle{\vspace{\droptitle}\centering\huge}
  \posttitle{\par}
  \author{Deependra Dhakal}
  \preauthor{\centering\large\emph}
  \postauthor{\par}
  \predate{\centering\large\emph}
  \postdate{\par}
  \date{April 10, 2018}

\usepackage{booktabs}
\usepackage{longtable}
\usepackage{array}
\usepackage{multirow}
\usepackage[table]{xcolor}
\usepackage{wrapfig}
\usepackage{float}
\usepackage{colortbl}
\usepackage{pdflscape}
\usepackage{tabu}
\usepackage{threeparttable}
\usepackage[normalem]{ulem}

\usepackage{float}

\begin{document}
\maketitle

{
\setcounter{tocdepth}{2}
\tableofcontents
}
There are three table obtained through \texttt{walk} here.

Here first, I'm going to reference first table \ref{tab:table-first},
the second follows \ref{tab:table-second}.

\begin{Shaded}
\begin{Highlighting}[]
\NormalTok{knitr}\OperatorTok{::}\KeywordTok{kable}\NormalTok{(}\KeywordTok{head}\NormalTok{(iris), }\DataTypeTok{caption =} \StringTok{"}\CharTok{\textbackslash{}\textbackslash{}}\StringTok{label\{tab:table-first\}Header data iris"}\NormalTok{, }\DataTypeTok{format =} \StringTok{"latex"}\NormalTok{, }\DataTypeTok{booktabs =} \OtherTok{TRUE}\NormalTok{) }\OperatorTok\StringTok{ }
\StringTok{  }\NormalTok{kableExtra}\OperatorTok{::}\KeywordTok{kable_styling}\NormalTok{(}\DataTypeTok{latex_options =} \KeywordTok{c}\NormalTok{(}\StringTok{"HOLD_position"}\NormalTok{))}
\end{Highlighting}
\end{Shaded}

\begin{table}[H]

\caption{\label{tab:unnamed-chunk-1}\label{tab:table-first}Header data iris}
\centering
\begin{tabular}[t]{rrrrl}
\toprule
Sepal.Length & Sepal.Width & Petal.Length & Petal.Width & Species\\
\midrule
5.1 & 3.5 & 1.4 & 0.2 & setosa\\
4.9 & 3.0 & 1.4 & 0.2 & setosa\\
4.7 & 3.2 & 1.3 & 0.2 & setosa\\
4.6 & 3.1 & 1.5 & 0.2 & setosa\\
5.0 & 3.6 & 1.4 & 0.2 & setosa\\
5.4 & 3.9 & 1.7 & 0.4 & setosa\\
\bottomrule
\end{tabular}
\end{table}

\begin{Shaded}
\begin{Highlighting}[]
\NormalTok{knitr}\OperatorTok{::}\KeywordTok{kable}\NormalTok{(}\KeywordTok{tail}\NormalTok{(iris), }\DataTypeTok{caption =} \StringTok{"}\CharTok{\textbackslash{}\textbackslash{}}\StringTok{label\{tab:table-second\}Tail data iris"}\NormalTok{, }\DataTypeTok{format =} \StringTok{"latex"}\NormalTok{, }\DataTypeTok{booktabs =} \OtherTok{TRUE}\NormalTok{) }\OperatorTok\StringTok{ }
\StringTok{  }\NormalTok{kableExtra}\OperatorTok{::}\KeywordTok{kable_styling}\NormalTok{(}\DataTypeTok{latex_options =} \KeywordTok{c}\NormalTok{(}\StringTok{"HOLD_position"}\NormalTok{))}
\end{Highlighting}
\end{Shaded}

\begin{table}[H]

\caption{\label{tab:unnamed-chunk-1}\label{tab:table-second}Tail data iris}
\centering
\begin{tabular}[t]{lrrrrl}
\toprule
  & Sepal.Length & Sepal.Width & Petal.Length & Petal.Width & Species\\
\midrule
145 & 6.7 & 3.3 & 5.7 & 2.5 & virginica\\
146 & 6.7 & 3.0 & 5.2 & 2.3 & virginica\\
147 & 6.3 & 2.5 & 5.0 & 1.9 & virginica\\
148 & 6.5 & 3.0 & 5.2 & 2.0 & virginica\\
149 & 6.2 & 3.4 & 5.4 & 2.3 & virginica\\
150 & 5.9 & 3.0 & 5.1 & 1.8 & virginica\\
\bottomrule
\end{tabular}
\end{table}

Let us refer to Cylinder number 4 table as Table \ref{tab:cyl4},
Cylinder number 6 table as Table \ref{tab:cyl6} and Cylinder number 8
table as Table \ref{tab:cyl8}.

\begin{Shaded}
\begin{Highlighting}[]
\NormalTok{mtcars }\OperatorTok\StringTok{ }
\StringTok{  }\KeywordTok{mutate}\NormalTok{(}\DataTypeTok{cyl =} \KeywordTok{factor}\NormalTok{(cyl)) }\OperatorTok\StringTok{ }
\StringTok{  }\KeywordTok{group_by}\NormalTok{(cyl) }\OperatorTok\StringTok{ }
\StringTok{  }\KeywordTok{do}\NormalTok{(}\DataTypeTok{model =} \KeywordTok{lm}\NormalTok{(mpg }\OperatorTok{~}\StringTok{ }\NormalTok{hp, .)) }\OperatorTok\StringTok{ }
\StringTok{  }\KeywordTok{mutate}\NormalTok{(}\DataTypeTok{rsqu =} \KeywordTok{summary}\NormalTok{(model)}\OperatorTok{$}\StringTok{`}\DataTypeTok{r.squared}\StringTok{`}\NormalTok{) }\OperatorTok\StringTok{ }
\StringTok{  }\NormalTok{broom}\OperatorTok{::}\KeywordTok{tidy}\NormalTok{(model) }\OperatorTok\StringTok{ }
\StringTok{  }\KeywordTok{split}\NormalTok{(}\DataTypeTok{f =}\NormalTok{ .}\OperatorTok{$}\NormalTok{cyl) }\OperatorTok\StringTok{ }
\StringTok{  }\KeywordTok{walk2}\NormalTok{(}\DataTypeTok{.y =} \KeywordTok{c}\NormalTok{(}\StringTok{"}\CharTok{\textbackslash{}\textbackslash{}}\StringTok{label\{tab:cyl4\}This is cylinder number 4"}\NormalTok{, }
               \StringTok{"}\CharTok{\textbackslash{}\textbackslash{}}\StringTok{label\{tab:cyl6\}This is cylinder number 6"}\NormalTok{, }
               \StringTok{"}\CharTok{\textbackslash{}\textbackslash{}}\StringTok{label\{tab:cyl8\}This is cylinder number 8"}\NormalTok{), }
        \DataTypeTok{.f =} \OperatorTok{~}\StringTok{ }\NormalTok{knitr}\OperatorTok{::}\KeywordTok{kable}\NormalTok{(.x, }\DataTypeTok{format =} \StringTok{"latex"}\NormalTok{, }
                            \DataTypeTok{booktabs =} \OtherTok{TRUE}\NormalTok{, }
                            \DataTypeTok{caption =}\NormalTok{ .y, }
                            \DataTypeTok{align =} \StringTok{"c"}\NormalTok{) }\OperatorTok\StringTok{ }
\StringTok{         }\NormalTok{kableExtra}\OperatorTok{::}\KeywordTok{kable_styling}\NormalTok{(}\DataTypeTok{position =} \StringTok{"center"}\NormalTok{, }\DataTypeTok{latex_options =} \StringTok{"HOLD_position"}\NormalTok{) }\OperatorTok\StringTok{ }
\StringTok{         }\KeywordTok{print}\NormalTok{())}
\end{Highlighting}
\end{Shaded}

\begin{table}[H]

\caption{\label{tab:three-tabs}\label{tab:cyl4}This is cylinder number 4}
\centering
\begin{tabular}[t]{ccccccc}
\toprule
cyl & rsqu & term & estimate & std.error & statistic & p.value\\
\midrule
4 & 0.2740558 & (Intercept) & 35.9830256 & 5.2012961 & 6.918088 & 0.0000693\\
4 & 0.2740558 & hp & -0.1127759 & 0.0611825 & -1.843271 & 0.0983986\\
\bottomrule
\end{tabular}
\end{table}\begin{table}[H]

\caption{\label{tab:three-tabs}\label{tab:cyl6}This is cylinder number 6}
\centering
\begin{tabular}[t]{ccccccc}
\toprule
cyl & rsqu & term & estimate & std.error & statistic & p.value\\
\midrule
6 & 0.0161462 & (Intercept) & 20.6738511 & 3.3044289 & 6.2564066 & 0.0015296\\
6 & 0.0161462 & hp & -0.0076133 & 0.0265776 & -0.2864543 & 0.7860202\\
\bottomrule
\end{tabular}
\end{table}\begin{table}[H]

\caption{\label{tab:three-tabs}\label{tab:cyl8}This is cylinder number 8}
\centering
\begin{tabular}[t]{ccccccc}
\toprule
cyl & rsqu & term & estimate & std.error & statistic & p.value\\
\midrule
8 & 0.0804492 & (Intercept) & 18.0800737 & 2.9875567 & 6.051793 & 0.0000574\\
8 & 0.0804492 & hp & -0.0142441 & 0.0139018 & -1.024622 & 0.3257538\\
\bottomrule
\end{tabular}
\end{table}

\begin{Shaded}
\begin{Highlighting}[]
\CommentTok{# mtcars %>% }
\CommentTok{#   group_by(cyl) %>% }
\CommentTok{#   do(model = lm(mpg ~ hp, .)) %>% }
\CommentTok{#   do(data.frame(}
\CommentTok{#   var = names(coef(.$model)),}
\CommentTok{#   coef(summary(.$model)))}
\CommentTok{# )}
\CommentTok{# }
\CommentTok{# models <- mtcars %>% }
\CommentTok{#   group_by(cyl) %>% }
\CommentTok{#   do(}
\CommentTok{#   mod_linear = lm(mpg ~ disp, data = .),}
\CommentTok{#   mod_quad = lm(mpg ~ poly(disp, 2), data = .)}
\CommentTok{# )}
\CommentTok{# }
\CommentTok{# models}
\CommentTok{# compare <- models %>% do(aov = anova(.$mod_linear, .$mod_quad))}
\CommentTok{# compare %>% mutate(p.value = aov$`Pr(>F)`[[2]])}
\end{Highlighting}
\end{Shaded}


\end{document}
