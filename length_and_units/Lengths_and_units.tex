\documentclass{article}


\begin{document}

\section*{Important tips and tricks:}
\begin{enumerate}
\item New paragraphs can be sepecified by: \\
$\verb|\\, \par, blank line|$ (recommended way)
\item Little extra space addition: \\
$I = \int e^x\, dx $ \hfil \checkmark \\
$I = \int e^x dx $ \hfil $\times$
\item Align (recommended way) vs eqnarray
\item Functions name: \\
$\log, \ln, \sin, \cos, \tan, \cosec^*, \exp$ \hfil \checkmark \\
$log, ln, sin, cos, tan, cosec, exp$ \hfil $\times$ \\
*declared math operator
\item Dots: \\
As mentioned\ldots \hfil \checkmark \\
As mentioned... \hfill $\times$
\item Page ranges: \\
--\hfil \checkmark \\
- \hfil $\times$

\item Avoid using these commands: \\
$\verb|\newline, \bigskip, \medskip, \smallskip,|$ \\
$\verb|\pagebreak, \linebreak|$

\item Cross referencing commands \\
$\verb|\ref{}, \eqref{}, \pageref{},|$ \\
$\verb|\cref{}, \Cref{}, \crefrange{}, \Crefrange{}|$

\item Using angle brackets \\
Never simply insert $\verb|<|$ while using inputenc package with utf8 encoding. Always use fontenc package with T1 fonts encoding if you feel the need to use angle brackets.

\item Inverted commas \\
For quoting some text, load $\verb|\usepackage{csquote}|$ package and use $\verb|\enquote{---}|$

\item Specifying blank lines
Out of $\verb|\\, \par, blank line|$, latter is the recommended style.

\item Getting verbatim
\noindent Prefixing macros with $\verb|\the|$ displays it's values. For example:

  \noindent baselineskip: \the\baselineskip \\
  % baselinewskip: \the\baselinewskip \\
  % baselinestretch: \the\baselinestretch \\
  columnsep: \the\columnsep \\
  columnwidth: \the\columnwidth \\
  unitlength: \the\unitlength \\
  linewidth: \the\linewidth \\
  textwidth: \the\textwidth \\
  textheight: \the\textheight \\
  evensidemargin: \the\evensidemargin \\
  oddsidemargin: \the\oddsidemargin \\
  paperwidth: \the\paperwidth \\
  paperheight: \the\paperheight \\
  parindent: \the\parindent \\
  parskip: \the\parskip \\
  tabcolsep: \the\tabcolsep \\
  topmargin: \the\topmargin
 
  % these codes all print a line paragraph that fills the entire line with "hrule" or "dot" between words.
   \noindent Deependra\hrulefill Dhakal \\
   \null \noindent Deependra\dotfill Dhakal \\
   \noindent Deependra\hrulefill Dhakal\null \\
  % this code makes a one-line paragraph that puts 'Name:' an inch from the right margin.
   \noindent\makebox[\linewidth]{\hspace{\fill}Name:\hspace{1in}}
 
 a\hfill b\vfill
 
 b\hfill a

\end{enumerate}

\end{document}