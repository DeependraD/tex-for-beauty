% This document keeps you settled getting a signature block done in a thesis or whateverelse

\documentclass{article}

% this first block of commands produce a \lsignblock command
% which is an automated way to get a complete left aligned signature block
% in single command stroke
%-------------------------------------------X
%Definition of right aligned signature block
%%%Defined the signature line using \leaders%%%
\def\mynewrule#1{\hbox to #1in{\leaders\hbox to 0.00625in{\hfil.\hfil}\hfill}}
%%%%Defined the signature line with \hfill%%%   
\def\lnewrule#1{\mynewrule{#1}}% {\hfill\mynewrule{#1}} for right alignment
%%%%Defined block to include rule and information%%%
\def\lsignblock#1#2#3#4#5{\lnewrule{#1}\par        
\vtop{\hsize=#1in\noindent{#2}\par\noindent{#3}% \hfill at front for right alignment
     \par\noindent{#4}\par\noindent{#5}}%
 \vskip1cm}
%-------------------------------------------X

% the wildcard command for two side-by-side boxes
%-------------------------------------------X
% \newcommand*\wildcard[2][5cm]{\vspace*{1cm}\parbox{#1}{\hrulefill\par#2}}
\newcommand*\wildcard[2][3cm]{\vspace*{1cm}\parbox{#1}{\centering\hrulefill\par#2\par}} % centered and extra version
%-------------------------------------------X


% the mysignrule
%-------------------------------------------X
\newcommand{\mysignrule}[1]{%
\vspace{5ex}
\rule{\linewidth}{0.5pt}\newline
#1\par
}
%-------------------------------------------X

\begin{document}

% demonstrate where the margin is
\hfill margin\par

% demonstration of \lsignblock:

\lsignblock{1}{Chairperson, Professor A}{}{}{}
\lsignblock{2}{Professor B}{}{}{}
\lsignblock{3}{Professor C}{}{}{}
\lsignblock{2}{Professor D}{Head of the Institute of Pearls}{50 Rainbow Street}{Succhitash, OH}

% demonstration of \wildcard command for side-by-side blocks

\begingroup
  \centering
  \wildcard{Name 1}
  \hspace{.5cm}
  \wildcard{Name 2}
  \hspace{.5cm}
  \wildcard{Name 3\\ Adding more might not be desirable}
  \par
\endgroup

% demonstration of a simple \parbox signature block:

\noindent \parbox{5cm}{
\underline{\hspace{4cm}}\\
Chairperson, Professor A\\[1cm]
\underline{\hspace{4cm}}\\
Professor B\\[1cm]
\underline{\hspace{4cm}}\\
Professor C\\[1cm]
}

% demonstration of \mysignrule

\begin{flushright}
\begin{minipage}{0.45\linewidth}
\mysignrule{Chairperson, Professor A}
\mysignrule{Professor B}
\mysignrule{Professor C}
\end{minipage}
\end{flushright}

% demonstration of a simple signature block (sufficient for most cases):

\noindent {\rule{0.2\textwidth}{.4pt}\\
Chairperson, Professor Shit\\
Direct from Hell\\[0.75cm]}

% demonstration of \hrulefill and \phantom to get signature block

Approved: \hrulefill

\hspace*{0mm}\phantom{Approved: }Fats Domino, Ph.D.

\hspace*{0mm}\phantom{Approved: }Chair of the Department of Nutrition\\[1cm]

% demonstration of two side by side blocks in tabular environment

\begin{tabular}{p{0.45\textwidth}cp{0.45\textwidth}}
  \cline{1-1} \cline{3-3} \\
  \centering {Name1\\ And his address and residence} & & \centering {Name2\\ And her address and residence and shitty little things whatever she does when he is not around}
\end{tabular}

% demonstration using \parbox

\vfill  % push the rest to the bottom of the page
\noindent 
\parbox[b]{0.4\linewidth}{% size of the first signature box
    \strut 
    Signed by: \\[2cm]% This 2cm is the space for the signature under the names
    % \hrule % replace with rule below to get a parwidth space
    \rule{0.6\linewidth}{.4pt}\\
    (Luke Skywalker)} 
\hspace{1cm} % distance between the two signature blocks 
\parbox[b]{0.4\linewidth}{% ...and the second one
    \strut 
    The force is strong in my family: \\[2cm]% This 2cm is the space for the signature under the names
    % \hrule % replace with rule below to get a parwidth space
    \rule{0.6\linewidth}{.4pt}\\
    (Darth Vader)} 
    \par\vspace{1cm} 


\end{document}